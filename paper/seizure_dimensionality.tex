\documentclass[11pt,letterpaper]{article}

% Packages
\usepackage[utf8]{inputenc}
\usepackage[T1]{fontenc}
\usepackage{amsmath,amssymb,amsfonts}
\usepackage{graphicx}
\usepackage{booktabs}
\usepackage{hyperref}
\usepackage{xcolor}
\usepackage{natbib}
\usepackage[margin=1in]{geometry}
\usepackage{caption}
\usepackage{subcaption}
\usepackage{float}

% Custom commands
\newcommand{\Deff}{D_{\text{eff}}}
\newcommand{\Dpr}{D_{\text{PR}}}
\newcommand{\Kth}[1]{K_{#1}}

% Title
\title{\textbf{Seizure as Critical Transition:\\Dimensional Signatures of Approaching Collapse in EEG}}

\author{
Ian Todd\\
\textit{Sydney Medical School, University of Sydney}\\
\texttt{itod2305@uni.sydney.edu.au}
}

\date{\today}

\begin{document}

\maketitle

\begin{abstract}
Seizures represent a critical phase transition in neural dynamics---from the flexible, multi-scale activity of healthy brain states to the hypersynchronous collapse of ictal discharge. We propose that the approach to seizure follows the signature pattern of critical transitions: \textbf{increasing coordination with decreasing variability}. Using the CHB-MIT scalp EEG database (subject chb01: 42 recordings, 7 seizures), we show that pre-ictal periods exhibit (1) \textit{higher} participation ratio $\Dpr$ than interictal baseline (5.69 vs.\ 5.05, $p < 10^{-28}$; non-overlapping windows, $n = 13,203$), indicating more coordinated oscillatory modes, but (2) dramatically \textit{reduced} variance (4.8$\times$ reduction in $\Dpr$ variance, 6.7$\times$ in $\lambda_1$ variance; Levene's test $p < 10^{-99}$), indicating the system has settled into a more stereotyped attractor. The ictal state then shows collapse: lower $\Dpr$ (5.04), steeper eigenvalue slope ($-3.19$ vs.\ $-2.76$), and fewer components required for variance thresholds ($\Kth{95} = 9.4$ vs.\ 10.6). This pattern---organization preceding collapse---mirrors critical phenomena in physics, where correlation length diverges before phase transitions. We propose that ``brain rate variability'' (the variance of dimensional metrics over time) may be a more sensitive seizure warning than the metrics themselves, analogous to how loss of heart rate variability predicts cardiac events. Classification of pre-ictal vs.\ interictal windows achieves AUC = 0.68 using dimensional features alone.
\end{abstract}

\textbf{Keywords:} seizure prediction, critical transition, EEG complexity, effective dimensionality, participation ratio, neural criticality

%==============================================================================
\section{Introduction}
%==============================================================================

\subsection{The Clinical Problem}

Epilepsy affects approximately 50 million people worldwide, with roughly one-third of patients experiencing seizures that remain uncontrolled despite medication \citep{who2019epilepsy}. The unpredictability of seizures is often more disabling than the seizures themselves---patients cannot drive, swim alone, or live without constant vigilance. A reliable early warning system would transform quality of life.

Decades of research have sought EEG biomarkers that predict seizures minutes to hours in advance. While many features show statistical differences between interictal (between-seizure) and pre-ictal (before-seizure) periods, clinical translation has been limited \citep{mormann2007seizure,kuhlmann2018seizure}. We argue this reflects a conceptual gap: most approaches seek features that \textit{decrease} before seizures, assuming pre-ictal states are ``simpler'' or ``more pathological.'' Our data suggest the opposite pattern.

\subsection{The Critical Transition Hypothesis}

Critical phase transitions in physical systems exhibit a characteristic signature: \textbf{increasing correlation with decreasing fluctuation}. As a system approaches a critical point:
\begin{itemize}
    \item Correlation length increases---distant parts of the system become more coordinated
    \item Fluctuations around the mean decrease---the system settles into a more stereotyped state
    \item Then the transition occurs---an avalanche, a phase change, a collapse
\end{itemize}

We propose that seizure follows this pattern:

\begin{quote}
\textbf{Seizure is a critical transition.} The pre-ictal state is not ``simpler'' than interictal---it is \textit{more organized}. Neural oscillators that normally operate semi-independently become increasingly coordinated. The system settles into a narrower attractor with reduced variability. This organization is the buildup; seizure is the avalanche.
\end{quote}

This reframes the warning signs. The signature of impending seizure is not ``things getting simpler'' but ``things getting too organized''---more coordination, less wandering, reduced variability in complexity metrics themselves.

\subsection{What is ``Dimensionality'' in EEG?}

Scalp EEG records voltage fluctuations from 20--256 electrodes. But these channels are not independent---nearby electrodes pick up overlapping sources, and distant regions often synchronize. The question is: \textit{how many independent patterns of activity are actually present?}

Consider an analogy: an orchestra has 100 musicians, but during a unison passage, there is effectively one ``mode'' of activity. During a complex fugue, there might be four independent voices. The \textbf{effective dimensionality} counts how many independent patterns contribute to the signal, regardless of how many sensors record it.

Mathematically, we compute the covariance matrix of the EEG channels and decompose it into eigenvalues $\lambda_1 \geq \lambda_2 \geq \ldots \geq \lambda_n$. Each eigenvalue represents the variance explained by one spatial pattern (eigenvector). If all variance concentrates in one mode ($\lambda_1$ dominates), the signal is effectively one-dimensional---all channels are doing the same thing. If variance spreads across many modes, the signal is high-dimensional---multiple independent patterns coexist.

The \textbf{participation ratio} $\Dpr$ summarizes this:
\begin{equation}
\Dpr = \frac{\left(\sum_i \lambda_i\right)^2}{\sum_i \lambda_i^2}
\end{equation}
This equals $n$ if all eigenvalues are equal (maximum dimensionality) and approaches 1 if one eigenvalue dominates (minimum dimensionality). For 23-channel EEG, $\Dpr$ typically ranges from 3--10 depending on brain state.

\subsection{Why Dimensionality Might Change Before Seizures}

Prior work has found that various ``complexity'' measures decrease before seizures---entropy drops, predictability increases, synchronization rises \citep{mormann2007seizure,lehnertz2017capturing}. The intuition is that the brain becomes ``simpler'' or ``more organized'' before the hypersynchronous discharge.

We hypothesized that effective dimensionality would follow the same pattern: fewer independent modes as the brain locks into a pre-seizure state. But the relationship turns out to be more subtle. The key insight is that we must track not just $\Dpr$ itself, but its \textbf{variance over time}:
\begin{itemize}
    \item High $\Dpr$ with high variance: healthy exploration of a large state space
    \item High $\Dpr$ with low variance: pathological organization, locked into a coherent attractor
    \item Low $\Dpr$ with low variance: collapse---the seizure itself
\end{itemize}

\subsection{Relation to Prior Work}

The ``criticality hypothesis'' in neuroscience proposes that healthy brain dynamics operate near a critical point, balancing order and disorder \citep{beggs2003neuronal,shew2013functional}. Our contribution is to show that the \textit{approach} to seizure exhibits the temporal signature of critical transitions: increasing coordination followed by collapse.

This also extends our broader ``coherence compute'' framework \citep{todd2025coherence}, which proposes that biological intelligence operates through high-dimensional continuous dynamics with discrete codes emerging at bottlenecks. Seizure, in this view, is \textit{uncontrolled} dimensional collapse---the system crossing a critical threshold globally rather than at regulated bottlenecks.

\subsection{Paper Structure}

Section~\ref{sec:methods} describes our dimensional profile metrics and analysis pipeline. Section~\ref{sec:results} presents results from the CHB-MIT database. Section~\ref{sec:theory} develops the theoretical framework connecting seizure to critical transitions. Section~\ref{sec:discussion} discusses clinical implications and limitations.


%==============================================================================
\section{Methods}
\label{sec:methods}
%==============================================================================

\subsection{Data: CHB-MIT Scalp EEG Database}

We analyzed subject chb01 from the CHB-MIT Scalp EEG Database \citep{shoeb2010application}, which contains continuous scalp EEG recordings from pediatric patients with intractable epilepsy:
\begin{itemize}
    \item 23 channels (International 10-20 bipolar montage)
    \item 256 Hz sampling rate
    \item 42 recordings totaling approximately 40 hours
    \item 7 expert-annotated seizures
\end{itemize}

\subsection{Window Classification}

For each recording, we defined temporal windows relative to seizures:
\begin{itemize}
    \item \textbf{Interictal}: Windows $>$1 hour from any seizure onset or offset
    \item \textbf{Pre-ictal}: Windows 0--30 minutes before seizure onset
    \item \textbf{Ictal}: Windows during annotated seizures
    \item \textbf{Postictal}: Windows 0--5 minutes after seizure offset
    \item \textbf{Excluded}: Windows 30--60 minutes before seizure (transition zone)
\end{itemize}

\subsection{Preprocessing}

Raw EEG was bandpass filtered (1--45 Hz, 4th-order Butterworth) to remove DC drift and high-frequency noise while preserving physiologically relevant oscillations (delta through low gamma). Data were segmented into 10-second windows with 1-second steps, yielding 131,584 analysis windows.

\subsection{Dimensional Profile Metrics}

For each window, we computed the channel covariance matrix $\mathbf{C} = \mathbf{X}^\top \mathbf{X} / (n-1)$ where $\mathbf{X}$ is the $T \times N$ matrix of filtered EEG (T timepoints, N channels). We extracted the eigenvalue spectrum $\lambda_1 \geq \lambda_2 \geq \cdots \geq \lambda_N$ and computed:

\subsubsection{Participation Ratio ($\Dpr$)}
\begin{equation}
\Dpr = \frac{\left(\sum_i \lambda_i\right)^2}{\sum_i \lambda_i^2}
\end{equation}
Measures the effective number of contributing modes. Range: 1 (single dominant mode) to $N$ (all modes equal).

\subsubsection{Variance Threshold Components ($\Kth{\theta}$)}
\begin{equation}
\Kth{\theta} = \min \left\{ k : \frac{\sum_{i=1}^{k} \lambda_i}{\sum_{i=1}^{N} \lambda_i} \geq \theta \right\}
\end{equation}
Minimum components to explain $\theta$\% of variance. We report $\Kth{80}$, $\Kth{90}$, $\Kth{95}$, $\Kth{99}$.

\subsubsection{Spectral Slope}
We fit $\log \lambda_i = \alpha - \beta \log i$ and report $\beta$ (more negative = steeper falloff).

\subsubsection{Dominant Mode Fraction ($\lambda_1$)}
\begin{equation}
f_1 = \lambda_1 / \sum_i \lambda_i
\end{equation}
Fraction of variance in the first principal component.

\subsection{Statistical Analysis}

We compared metrics across window classes using Welch's t-test (unequal variances) and computed Cohen's $d$ effect sizes. Variance differences were tested with Levene's test. Classification used logistic regression with balanced class weights, evaluated via 5-fold cross-validation AUC.


%==============================================================================
\section{Results}
\label{sec:results}
%==============================================================================

\subsection{The Surprise: Pre-ictal Dimensionality is Higher, Not Lower}

Table~\ref{tab:metrics} presents dimensional profile metrics by window class. The key finding is counterintuitive: \textbf{pre-ictal periods show \textit{higher} participation ratio than interictal baseline}, not lower.

\begin{table}[h]
\centering
\caption{Dimensional profile metrics by window class (mean $\pm$ SD). Pre-ictal shows higher $\Dpr$ but dramatically lower variance than interictal. Ictal shows collapse.}
\label{tab:metrics}
\begin{tabular}{lcccc}
\toprule
Metric & Interictal & Pre-ictal & Ictal & Effect size $d$ \\
 & ($n = 121,748$) & ($n = 9,836$) & ($n = 449$) & (pre vs.\ inter) \\
\midrule
$\Dpr$ & $5.05 \pm 1.80$ & $5.69 \pm 0.81$ & $5.04 \pm 0.75$ & $+0.37$ \\
$\Kth{95}$ & $10.53 \pm 1.44$ & $10.57 \pm 0.68$ & $9.36 \pm 0.95$ & $+0.02$ \\
$\Kth{99}$ & $15.27 \pm 1.00$ & $15.24 \pm 0.56$ & $14.12 \pm 0.85$ & $-0.03$ \\
Slope $\beta$ & $-2.77 \pm 0.20$ & $-2.76 \pm 0.14$ & $-3.19 \pm 0.13$ & $+0.03$ \\
$\lambda_1$ & $0.395 \pm 0.141$ & $0.326 \pm 0.053$ & $0.343 \pm 0.048$ & $-0.50$ \\
\bottomrule
\end{tabular}
\end{table}

Because overlapping windows inflate effective sample size, we verified all comparisons using non-overlapping windows (every 10th window, $n_{\text{inter}} = 12,175$, $n_{\text{pre}} = 984$). Effects remain highly significant: $\Dpr$ $t = -11.1$, $p < 10^{-28}$; $\lambda_1$ $t = 15.0$, $p < 10^{-50}$; variance ratios 4.8$\times$ and 6.7$\times$ respectively (Levene $p < 10^{-99}$). The pattern is robust:
\begin{itemize}
    \item $\Dpr$ is \textit{higher} in pre-ictal ($d = +0.37$)
    \item $\lambda_1$ (dominant mode fraction) is \textit{lower} in pre-ictal ($d = -0.50$)
    \item $\Kth{95}$ and slope are essentially unchanged
\end{itemize}

This initially seems to contradict a ``collapse'' hypothesis. But the variance tells the real story.

\subsection{The Key Finding: Variance Reduction}

Table~\ref{tab:variance} shows the dramatic reduction in metric variance from interictal to pre-ictal.

\begin{table}[h]
\centering
\caption{Variance reduction from interictal to pre-ictal. The system becomes dramatically more stereotyped before seizure.}
\label{tab:variance}
\begin{tabular}{lccc}
\toprule
Metric & Interictal Var & Pre-ictal Var & Reduction \\
\midrule
$\Dpr$ & 3.23 & 0.66 & $4.9\times$ \\
$\lambda_1$ & 0.0199 & 0.0028 & $7.0\times$ \\
$\Kth{95}$ & 2.07 & 0.46 & $4.5\times$ \\
\bottomrule
\end{tabular}
\end{table}

All variance comparisons are highly significant (Levene's test $p < 10^{-100}$). The interpretation:
\begin{itemize}
    \item \textbf{Interictal}: The system wanders through a large state space. Sometimes $\Dpr$ is high (many coordinated modes), sometimes low (few active modes). High variance reflects healthy exploration.
    \item \textbf{Pre-ictal}: The system locks into an organized attractor. $\Dpr$ is consistently elevated---\textit{more} modes are coordinated---but with little variation. The system has stopped exploring.
    \item \textbf{Ictal}: Collapse. $\Dpr$ drops, slope steepens dramatically ($-3.19$ vs.\ $-2.76$), $\Kth{95}$ decreases. All variance concentrated in hypersynchronous modes.
\end{itemize}

\subsection{The Ictal Collapse}

The transition from pre-ictal to ictal shows unambiguous collapse (Table~\ref{tab:ictal}):

\begin{table}[h]
\centering
\caption{Pre-ictal to ictal transition: dimensional collapse.}
\label{tab:ictal}
\begin{tabular}{lccc}
\toprule
Metric & Pre-ictal & Ictal & Effect size $d$ \\
\midrule
$\Dpr$ & 5.69 & 5.04 & $-0.81$ \\
$\Kth{80}$ & 5.69 & 4.77 & $-1.54$ \\
$\Kth{95}$ & 10.57 & 9.36 & $-1.74$ \\
$\Kth{99}$ & 15.24 & 14.12 & $-1.95$ \\
Slope $\beta$ & $-2.76$ & $-3.19$ & $-3.08$ \\
\bottomrule
\end{tabular}
\end{table}

Effect sizes are large to very large ($|d| > 0.8$ for all metrics). The eigenvalue spectrum steepens dramatically (slope effect size $d = -3.08$), indicating variance concentrating into fewer modes---the signature of hypersynchronous collapse.

\subsection{Classification Performance}

Logistic regression classification of pre-ictal vs.\ interictal using single-window dimensional features achieved modest performance:
\begin{itemize}
    \item Baseline (single-window): AUC = $0.68 \pm 0.05$
\end{itemize}

However, adding temporal variance features (``Brain Rate Variability'') dramatically improves classification:
\begin{itemize}
    \item With BRV features: AUC = $0.83 \pm 0.14$
    \item BRV features only: AUC = $0.82 \pm 0.19$
\end{itemize}

The variance features alone (BRV$_{\Dpr}$ and BRV$_{\lambda_1}$ computed over 5-minute windows) achieve higher AUC than the raw metrics, confirming that the discriminative information lies in \textit{variance reduction} rather than mean shifts. This validates the ``Brain Rate Variability'' concept: tracking how much the system wanders is more informative than tracking where it is.

\subsection{Robustness: Non-Overlapping Windows}

Our analysis used overlapping windows (10s duration, 1s step), introducing autocorrelation that inflates effective sample size. To address this, we repeated key analyses using only non-overlapping windows (every 10th window), yielding $n = 12,175$ interictal and $n = 984$ pre-ictal independent samples.

Results remain highly significant:
\begin{itemize}
    \item $\Dpr$: interictal = 5.05, pre-ictal = 5.69, $p = 1.5 \times 10^{-28}$
    \item $\lambda_1$: interictal = 0.395, pre-ictal = 0.327, $p = 1.5 \times 10^{-50}$
    \item Variance ratios: 4.8$\times$ ($\Dpr$), 6.7$\times$ ($\lambda_1$), Levene $p < 10^{-99}$
\end{itemize}

The effect is robust to autocorrelation correction.

\subsection{Transition Zone Gradient}

We analyzed the previously excluded 30--60 minute window to test for a gradient approaching seizure:

\begin{table}[h]
\centering
\caption{Transition zone: metrics show gradient approaching seizure.}
\label{tab:gradient}
\begin{tabular}{lcccc}
\toprule
Time (min) & $n$ & $\Dpr$ mean & $\Dpr$ var & $\lambda_1$ mean \\
\midrule
$-60$ to $-45$ & 12,222 & 4.92 & 2.78 & 0.400 \\
$-45$ to $-30$ & 39,594 & 5.10 & 2.87 & 0.388 \\
$-30$ to $-15$ & 171,062 & 5.09 & 2.81 & 0.387 \\
$-15$ to $0$ & 234,422 & 5.18 & 2.84 & 0.381 \\
\bottomrule
\end{tabular}
\end{table}

The pattern shows progressive increase in $\Dpr$ and decrease in $\lambda_1$ as seizure approaches, consistent with increasing network coordination before collapse.

\subsection{Temporal Dynamics}

Figure~\ref{fig:timecourse} shows dimensional metrics aligned to seizure onset across all 7 seizures. The pattern varies by seizure but consistently shows:
\begin{itemize}
    \item Elevated, stable $\Dpr$ in the 30--60 minutes before seizure
    \item Sharp drop at seizure onset
    \item Recovery in postictal period
\end{itemize}

\begin{figure}[H]
\centering
\includegraphics[width=\textwidth]{figures/fig5_chb01_seizure_aligned.pdf}
\caption{\textbf{Seizure-aligned dimensional dynamics.} Mean $\pm$ SD of dimensional metrics across 7 seizures, aligned to onset (red dashed line). Note the stable, elevated $\Dpr$ before seizure followed by sharp drop at onset.}
\label{fig:timecourse}
\end{figure}


%==============================================================================
\section{Theoretical Framework}
\label{sec:theory}
%==============================================================================

\subsection{Critical Transitions in Physical Systems}

Critical phase transitions exhibit universal signatures \citep{scheffer2009early}:
\begin{enumerate}
    \item \textbf{Critical slowing down}: The system takes longer to recover from perturbations
    \item \textbf{Increased autocorrelation}: States become more persistent
    \item \textbf{Increased variance} (in some systems) or \textbf{decreased variance} (in systems approaching a fixed point)
    \item \textbf{Flickering}: Occasional excursions toward the alternative state
\end{enumerate}

Our data suggest that pre-ictal dynamics exhibit a specific pattern: \textbf{increased coordination (higher $\Dpr$) with decreased variability (lower variance)}. This is consistent with approach to a \textit{fixed point} or narrow attractor---the system settles into an organized state before the catastrophic transition.

\textbf{Note on terminology:} In classical critical slowing down (e.g., Scheffer et al., ecological tipping points), variance often \textit{increases} as the system ``flickers'' near a bifurcation \citep{scheffer2009early}. Our observation of \textit{decreased} variance suggests a different mechanism: ``dynamical rigidification'' or ``attractor deepening,'' where the system becomes increasingly trapped in an organized basin before catastrophic escape. The pre-ictal brain is not loosening toward chaos but tightening toward pathological order.

\subsection{The Sandpile Analogy}

Consider a sandpile approaching criticality:
\begin{itemize}
    \item \textbf{Subcritical}: Sand grains land randomly, local avalanches occur, the pile has variable local slopes
    \item \textbf{Near-critical}: The pile steepens. Local regions become coordinated---a grain added here affects grains there. The slope becomes more uniform (lower variance) but steeper (more organized)
    \item \textbf{Critical}: Avalanche. A single grain triggers global collapse
\end{itemize}

The analogy to seizure:
\begin{itemize}
    \item \textbf{Interictal}: Neural oscillators operate semi-independently. Some are synchronized, some not. High variability in coordination metrics.
    \item \textbf{Pre-ictal}: Oscillators become increasingly coupled. More coordinated activity (higher $\Dpr$) but locked into a stereotyped pattern (lower variance). The system is ``steepening.''
    \item \textbf{Ictal}: Global synchronization. All variance collapses into hypersynchronous discharge. The avalanche.
\end{itemize}

\subsection{Why Higher $\Dpr$ Can Indicate Pathology}

The participation ratio counts ``effective dimensions''---how many modes contribute substantially to variance. Counterintuitively, \textit{more} coordinated activity can produce \textit{higher} $\Dpr$:

\begin{itemize}
    \item In healthy interictal states, some brain regions are active (high eigenvalues) while others are quiet (low eigenvalues). The eigenvalue spectrum is heterogeneous.
    \item In pre-ictal states, previously quiet regions become entrained. More oscillators contribute to the covariance structure. The spectrum becomes more uniform---\textit{more} modes above threshold.
    \item In ictal states, all regions lock to one pattern. The spectrum collapses to a single dominant mode.
\end{itemize}

The trajectory is: heterogeneous $\to$ uniformly coordinated $\to$ collapsed.

Figure~\ref{fig:spectra} illustrates this progression using reconstructed eigenvalue spectra from the mean $\lambda_1$ and slope for each state.

\begin{figure}[H]
\centering
\includegraphics[width=0.7\textwidth]{figures/fig6_eigenvalue_spectra.pdf}
\caption{\textbf{Eigenvalue spectra by state.} Reconstructed mean eigenvalue spectra (log scale) for interictal, pre-ictal, and ictal windows. Interictal shows heterogeneous spectrum; pre-ictal shows more uniform mid-range modes; ictal shows steep falloff (collapse). $\beta$ = log-log slope.}
\label{fig:spectra}
\end{figure}

\subsection{Brain Rate Variability}

Heart rate variability (HRV) predicts cardiac events because \textit{reduced variability} indicates loss of adaptive flexibility \citep{malik1996heart}. A healthy heart varies its rate in response to demands; a failing heart locks into a narrow pattern.

We propose an analogous ``brain rate variability'' (BRV): the variance of dimensional metrics over time. Our data suggest:
\begin{itemize}
    \item High BRV (high variance in $\Dpr$, $\lambda_1$, etc.) indicates healthy exploration
    \item Low BRV indicates the system has locked into an attractor---a warning sign
    \item The specific attractor matters less than the loss of flexibility
\end{itemize}

This reframes seizure prediction: track not just ``is $\Dpr$ high or low?'' but ``has $\Dpr$ variance dropped?'' The 5--7$\times$ variance reduction we observe is a stronger signal than the modest shift in means.

Figure~\ref{fig:brv} shows the BRV time course aligned to seizure onset. Note that both mean $\Dpr$ and its variance (BRV) show characteristic changes in the 30 minutes preceding seizure.

\begin{figure}[H]
\centering
\includegraphics[width=\textwidth]{figures/fig7_brv_timecourse.pdf}
\caption{\textbf{Brain Rate Variability around seizures.} (a) Mean $\Dpr$ remains elevated and stable in the pre-ictal period. (b) BRV (variance of $\Dpr$ over 5-min windows) drops before seizure onset. (c) BRV of $\lambda_1$ shows similar reduction. (d) Summary statistics comparing baseline ($<-30$ min) to pre-seizure ($-30$ to 0 min) windows.}
\label{fig:brv}
\end{figure}

\subsection{Connection to Coherence Compute}

In our broader framework \citep{todd2025coherence}, biological intelligence operates through high-dimensional continuous dynamics with discrete codes emerging at dimensional bottlenecks. Effective dimensionality $\Deff$ is modulated by oscillatory control (beta/gamma balance).

Seizure, in this view, is \textit{uncontrolled} critical transition:
\begin{itemize}
    \item Normal cognition involves controlled dimensional reduction at specific bottlenecks (attention, memory encoding, motor output)
    \item Seizure involves \textit{global} loss of dimensional control
    \item The pre-ictal buildup is the system approaching criticality network-wide
    \item The ictal discharge is the avalanche
\end{itemize}


%==============================================================================
\section{Discussion}
\label{sec:discussion}
%==============================================================================

\subsection{Clinical Implications}

\textbf{Reframing prediction targets:} Rather than seeking metrics that are ``abnormal'' in pre-ictal periods, we should track \textit{variability reduction}. A patient whose $\Dpr$ variance drops by 50\% over an hour may be approaching seizure even if the mean $\Dpr$ is unremarkable.

\textbf{Closed-loop intervention:} Current responsive neurostimulation devices (e.g., NeuroPace RNS) trigger on ictal patterns. Earlier intervention---triggered by variance reduction---might abort seizures during the ``steepening'' phase before the avalanche begins.

\textbf{Multi-timescale monitoring:} Short-term variance (minutes) may predict imminent seizures. Long-term variance (days) may indicate overall seizure susceptibility or medication efficacy.

\subsection{Validation via Synthetic Model}

To confirm that our eigenvalue-based metrics capture the intended phenomena, we constructed a synthetic oscillator model where we explicitly control the latent dimensionality (Supplementary Materials). The model transitions from high-dimensional (many independent oscillators) through medium-dimensional (partial synchronization) to low-dimensional (full synchronization). The dimensional profile metrics ($\Dpr$, $\lambda_1$, slope) track these transitions exactly as predicted, with the ``pre-collapse'' phase showing elevated $\Dpr$ and reduced variance before the final collapse. This provides interpretive validation that eigenvalue spectra reliably indicate underlying coordination structure.

\subsection{Relation to Existing Literature}

Our findings connect to several established lines of research:

\textbf{Critical brain hypothesis:} The brain is proposed to operate near criticality for optimal information processing \citep{beggs2003neuronal,shew2013functional}. Our data suggest that seizure represents departure from healthy criticality---not ``too much chaos'' but ``too much order.''

\textbf{Entropy measures:} Studies using sample entropy, permutation entropy, and multiscale entropy often find \textit{decreased} complexity before seizures \citep{gao2015multiscale}. This is consistent with our variance reduction finding: both reflect the system locking into a stereotyped state.

\textbf{Synchronization measures:} Increased phase synchronization before seizures is well-documented \citep{mormann2007seizure}. Our $\Dpr$ increase is consistent---more synchronized oscillators produce more coherent modes in the covariance spectrum.

\subsection{Limitations}

\textbf{Single subject:} We analyzed one patient (chb01) in depth. While the findings are statistically robust (n $>$ 130,000 windows), replication across subjects is essential. Patient heterogeneity in epilepsy syndromes may affect generalizability.

\textbf{Scalp EEG resolution:} Scalp recordings are spatial mixtures of cortical sources. The eigenvalue spectrum reflects this mixture, not the true cortical geometry. Intracranial recordings would provide cleaner estimates.

\textbf{Pre-ictal definition:} We defined pre-ictal as 0--30 minutes before seizure. The optimal window may vary by patient and seizure type. The 30--60 minute ``transition zone'' we excluded deserves further study.

\textbf{Classification performance:} AUC = 0.68 is moderate. This reflects the challenge that pre-ictal states are not obviously abnormal window-by-window. The discriminative information is in temporal patterns (variance reduction) that require tracking over time.

\textbf{Base rate problem in deployment:} Our analysis uses data from patients \textit{known} to have seizures, with labeled pre-ictal windows. This sidesteps a fundamental challenge for real-world deployment: the overwhelming majority of time---even for epilepsy patients---is seizure-free. A classifier with 95\% sensitivity and 95\% specificity sounds excellent, but if the base rate of ``seizure imminent'' is 0.1\%, the positive predictive value collapses to $\sim$2\%. This is the same ``cascade'' problem that plagues screening programs and diagnostic testing generally: rare events require extraordinary specificity to avoid false positive burden. Continuous seizure monitoring cannot simply be ``wire everyone up to the seizure alarm.'' Practical deployment likely requires: (1) restricting monitoring to high-risk populations where base rates are higher; (2) using alerts as \textit{warnings} triggering closer observation rather than immediate intervention; or (3) combining EEG variance with other modalities (HRV, sleep state, medication adherence) to improve specificity. Our BRV findings suggest that \textit{temporal patterns}---sustained variance reduction over tens of minutes---may provide better specificity than instantaneous classification, but validation under realistic base rates remains essential.

\subsection{Future Directions}

\textbf{Multi-subject validation:} Extend analysis to all 24 CHB-MIT subjects and to adult cohorts (e.g., TUH EEG corpus).

\textbf{Variance-based prediction:} Develop algorithms that explicitly track variance reduction as a warning signal, not just metric values.

\textbf{Intracranial validation:} Test whether the criticality pattern is visible in local field potentials from implanted electrodes.

\textbf{Stress/inflammation integration:} Combine EEG dimensional variance with HRV and inflammatory markers for multi-modal seizure risk prediction.


%==============================================================================
\section{Conclusion}
%==============================================================================

We have shown that the approach to seizure exhibits the signature of a critical transition: \textbf{increasing coordination with decreasing variability}. Pre-ictal periods show elevated participation ratio (more coordinated modes) but dramatically reduced variance (the system locked into an organized attractor). Ictal discharge is the collapse---eigenvalue spectra steepening, variance concentrating into hypersynchronous modes.

This reframes seizure prediction. The warning sign is not ``the brain getting simpler'' but ``the brain getting too organized''---loss of the healthy variability that indicates flexible, adaptive dynamics. Tracking ``brain rate variability'' (variance in dimensional metrics over time) may provide earlier warning than tracking the metrics themselves.

More broadly, this connects clinical epileptology to the physics of critical transitions. Seizure is not merely hyperexcitability but a phase transition---the neural equivalent of an avalanche in a system that has steepened beyond the critical point. Understanding seizure as criticality may open new approaches to prediction, intervention, and treatment.


%==============================================================================
\section*{Acknowledgments}
%==============================================================================

The author thanks the creators of the CHB-MIT database for making their data publicly available. This work was conducted independently without institutional funding.

%==============================================================================
\section*{Data and Code Availability}
%==============================================================================

All analysis code is available at: \url{https://github.com/todd866/seizure-dynamics}

The CHB-MIT database is available at: \url{https://physionet.org/content/chbmit/1.0.0/}


\bibliographystyle{plainnat}
\bibliography{references}

\end{document}
